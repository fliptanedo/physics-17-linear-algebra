%!TEX root = paper.tex
%% Update the above with the appropriate root

\section{Code example}

These are examples of the \texttt{listings} package for typesetting code. See Overleaf\footnote{\url{https://www.overleaf.com/learn/latex/Code_listing}} and tex.stackexchange\footnote{\url{https://tex.stackexchange.com/a/350242}} for discusisons.

\subsection{Basic Example}
\begin{pyin}
print("Hello world")
\end{pyin}

\begin{pyprint}
Hello world
\end{pyprint}

\subsection{Outputs and returned values}
And here we also have a return value in the last line of the input cell.
\begin{pyin}[labelOfTheSecondInput]
def twicify(arg):
    print("Received:", arg, "- Will double now...")
    return 2 * arg
twicify(1)
\end{pyin}

\begin{pyprint}
Received: 1 - Will double now...
\end{pyprint}

\begin{pyout}
2
\end{pyout}

\subsection{Referencing input}
You can also reference the labeled input \ref{labelOfTheSecondInput}, from above.
\begin{pyin}
"and the counter will automatically do the right thing :)"
\end{pyin}
\begin{pyout}
'and the counter will automatically do the right thing :)'
\end{pyout}
